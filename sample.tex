\documentclass[11pt,aspectratio=169]{beamer}

\usepackage{amssymb}

\title{Torus amplitudes and modular invariance}
\date[16.5.2022]{Seminar on Theoretical Physics}
\author{Otto T.P. Schmidt}
%\institute[Organizational Unit]{Organizational Unit\\can spread over 2 lines}
\institute[Department of Physics]{}
\usetheme{eth}

\colorlet{titlefgcolor}{ETHblue}
\colorlet{accentcolor}{ETHred}

\begin{document}

%\def\titlefigure{elements/title-page-image}		% Default image
%\def\titlefigure{elements/title-page-image-43}	% Use this for 4:3 presentations

\titleframe

% \colorlet{titlefgcolor}{ETHpurple}
% \def\titlefigure{elements/title-page-image-alt}
% \title{Different background}
% \titleframe

% \colorlet{titlefgcolor}{ETHgreen}
% \colorlet{titlebgcolor}{ETHgreen!60!black}
% \def\titlefigure{}
% \setlength{\titleboxwidth}{0.75\textwidth}			% Change box width
% \title{Or even a plain color, especially if your title is very long and leaves no space for what's behind the colored box}
% \titleframe

\tocframe

\section{Motivation}

\begin{frame}[fragile]{\underline{Interactions and observables}}

	In the study of string interactions, the ultimate goal will be the assignment of a probability for a certain process and the prediction of a physical cross section.
	\\~\\
	As outlined in Section 22, the computation of an observable cross section involves a series of steps:
	\begin{enumerate}
		\item Canonical representation of string diagram through moduli space
		\item Compute scattering amplitude by means of conformal field theory
		\item Convert scattering amplitude into a cross section
	\end{enumerate}
	
	
		




\end{frame}

\begin{frame}[fragile]{\underline{Loop amplitudes in string theory}}
	
	In order to obtain accurate scattering amplitudes of processes,
	one needs to include contributions from loops in string diagrams. 
	\\~\\
	These loops can be seen as contributions
	from the next higher order pertubation. 
	Graphically we consider the following processes:
	\begin{figure}[htbp]
		\centering
		\includegraphics[width = 0.55\textwidth]{elements/Feynman loop}
	\end{figure}


\end{frame}

\begin{frame}{\underline{Ultraviolet divergence}}

	Amplitudes from virtual processes as depicted before can lead to ultraviolet (UV) divergences in quantum field theory (QFT).
	\\~\\
	Whereas QFT must employ complex renormalizations to deal with these UV divergences, we do not encounter these problems in string theory.
	
\end{frame}



\section{The moduli space of tori}

\subsection{One-loop open strings}

\begin{frame}[fragile]{\underline{One-loop open strings}}

	Before approaching the moduli space of tori, lets consider a one-loop open string with light-cone momentum $p^{+}$. This will serve as an intuitive analogon.

	The light-cone diagram is:
	\begin{figure}[htbp]
		\centering
		\includegraphics[width = 0.45\textwidth]{elements/one-loop open string.PNG}
	\end{figure}

	For fixed external momentum $p^+$ we find the two parameters: $\Delta T \in (0, \infty)$ and $p_{1}^{+} \in (0, p^+)$.
	\\
	$\rightarrow$ The class of Riemann surfaces of this process has two moduli.

\end{frame}

\begin{frame}{\underline{Canonical annulus}}

	Use $w = \tau + i \sigma$ and apply conformal transformations:
	\\~\\
	\begin{enumerate}
		\item Exponential map: $z = exp[\frac{w}{2 \alpha^{'} p^+}]$
		\item Linear fractional transformation: $\eta = \frac{1+iz}{1-iz}$
		% \item Canonical annulus: \textit{A region in} \mathbb{C} \textit{that is topologically an annulus can be mapped conformally to a canonical annulus}
		\item Canonical annulus: \textit{A region in $\mathbb{C}$ that is topologically an annulus can be mapped conformally to a canonical annulus}

	\end{enumerate}

	\begin{columns}
		\begin{column}{0.25\textwidth}
			\includegraphics[width=\columnwidth]{elements/exponential map.PNG}
		\end{column}
		\begin{column}{0.25\textwidth}
			\includegraphics[width=\columnwidth]{elements/LFT.PNG}
		\end{column}
		\begin{column}{0.25\textwidth}
			\includegraphics[width=\columnwidth]{elements/canonical annulus.PNG}
		\end{column}
	\end{columns}
	
\end{frame}

\subsection{Rectangular tori}

\begin{frame}{\underline{Rectangular tori}}

	In order to apply the concept of moduli spaces to a torus, we need to assure that a torus is indeed a Riemann surface.
	\\
	Consider a rectangular region of $\mathbb{C}$. By applying the analytic identifications $z \sim z + L_1$ and $z \sim z + iL_2$
	we obtain a torus. This shows that the region remains a Riemann surface. 
	\\
	Graphically:
	\begin{columns}
		\begin{column}{0.25\textwidth}
			\includegraphics[width=\columnwidth]{elements/region C.PNG}
		\end{column}
		\begin{column}{0.2\textwidth}
			\includegraphics[width=\columnwidth]{elements/first ident.PNG}
		\end{column}
		\begin{column}{0.25\textwidth}
			\includegraphics[width=\columnwidth]{elements/second ident.PNG}
		\end{column}
	\end{columns}


	
\end{frame}

\begin{frame}{\underline{Parametrisation}}

	We have:
	\\
	\begin{block}{Rectangular torus}
		$z \sim z + L_1$ and $z \sim z + iL_2$
	\end{block}
	\bigskip
	By applying $z^{'} = \frac{z}{L_1}$ the identifications become:
	\begin{block}{Torus parameter T}
		$z^{'} \sim z^{'} + 1$ and $z^{'} \sim z^{'} + iT$ with $T = \frac{L_2}{L_1}$
	\end{block}

\end{frame}

\begin{frame}{\underline{Ultraviolet divergence}}

	T is a parameter of the torus but does not yet define the moduli space, i.e. tori with different T can be conformally equivalent.
	\\
	Consider the following series of conformal maps to a rectangular torus with $T<1$:

	\begin{figure}[htbp]
		\centering
		\includegraphics[width = 0.30\textwidth]{elements/three maps.PNG}
	\end{figure}
	\begin{block}{Rectangular torus}
		Tori with $T$ and $\frac{1}{T}$ are conformally equivalent
		\\
		$\rightarrow$ The moduli space can be chosen to be $T \in (0,1]$ or $T \in [1, \infty)$
	\end{block}
	
\end{frame}

\subsection{General tori}

\begin{frame}{\underline{General tori}}
	Rectangular tori represent only a subset off all conformally inequivalent tori. Let's construct a more general class of tori:
	\\
	\begin{block}{General construction of a torus Riemann surface}
		Choose $\omega_1, \omega_2 \in \mathbb{C}$ with $\Im(\frac{\omega_1}{\omega_2})$.
		\\
		A torus is obtained by the indentifications $z \sim z + \omega_1$ and $z \sim z + \omega_2$.
		\\
		By scaling we obtain $z \sim z + 1$ and $z \sim z + \tau$ with $\tau = \frac{\omega_2}{\omega_1}$, $\Im(\tau) >0$.
	\end{block}
	$\rightarrow$ Note that for $\tau = iT$ ($\Leftrightarrow \Re(\tau) = 0$) we consider the rectangular torus.
	\begin{figure}[htbp]
		\centering
		\includegraphics[width = 0.45\textwidth]{elements/general torus.PNG}
	\end{figure}
\end{frame}

\begin{frame}{\underline{Twisting the torus}}
	Intuitively, if a cylinder is twisted and the end surfaces are connected, we expect a different torus. 
	\\
	\underline{Formally}: Consider $\Re(\tau) \neq 0$ and a point $P = 0 = \tau$. We can reconstruct the rectangular $\textit{fundamental domain}$ by using the identification $z \sim z + 1$.
	\\
	\underline{Graphically}:
	\begin{figure}[htbp]
		\centering
		\includegraphics[width = 0.55\textwidth]{elements/twisted torus.PNG}
	\end{figure}
	
\end{frame}

\begin{frame}{\underline{Twisting parameter}}

	
	The point $P$ is no longer identified with a point on the perpendicular. Indeed the degree of twisting is parametrised by $\theta = 2\pi \Re(\tau)$.
	How does the twisting angle $\theta$ affect the torus parameter $\tau$?
	\\~\\
	Consider the map $\tau \rightarrow \tau + 1$:
	\begin{figure}[htbp]
		\centering
		\includegraphics[width = 0.55\textwidth]{elements/tau map.PNG}
	\end{figure}
	With the identification $z \sim z + 1$ we can conclude $\tau \sim \tau + 1$ and hence $\theta \sim \theta + 2\pi$.
	\begin{block}{Note:}
		The "twisting" does not correspond to actual torsion. It is the mere identification of the points $P$.
	\end{block}
\end{frame}


\begin{frame}{\underline{Moduli space I}}
	So far we have established that $\Im(\tau) > 0 \Leftrightarrow \tau \in \mathbb{H}$.
	However, the identification $\tau \sim \tau + 1$ implies that the space of inequivalent tori is smaller. Indeed:
	\begin{block}{Strip $\mathcal{S}_{0}$}
		$\mathcal{S}_{0} = \{-\frac{1}{2} < \Re(\tau) < \frac{1}{2}, \Im(\tau) > 0\}$
	\end{block}
	Is $\mathcal{S}_{0}$ the moduli space of tori?
	\\
	For rectangular tori we found that $T, \frac{1}{T}$ yield equivalent tori. Since $\tau = iT$ we have $\tau' = \frac{i}{T} = -\frac{1}{\tau}$.
	\begin{figure}[htbp]
		\centering
		\includegraphics[width = 0.55\textwidth]{elements/s module trans.PNG}
	\end{figure}
\end{frame}

\begin{frame}{\underline{Moduli space II}}

	We have found two identifications:
	\begin{block}{T-module transform}
		$\tau \sim \tau + 1$
	\end{block}
	\begin{block}{S-module transform}
		$\tau \sim -\frac{1}{\tau}$
	\end{block}
	The corresponding fundamental domain should be a subset of $\mathcal{S}_{0}$. Indeed, the S-module transform identifies points in $|\tau| < 1$ with points in $|\tau| > 1$.
	\\
	Therefore:
	\begin{block}{Fundamental domain $\mathcal{F}_{0}$}
		$\mathcal{F}_{0} = \{-\frac{1}{2} < \Re(\tau) < \frac{1}{2}, \Im(\tau) > 0, |\tau| \geq 1 \textrm{ and } \Re(\tau) \geq 0 \textrm{ if } |\tau| = 1\}$
	\end{block}
	
\end{frame}


\begin{frame}

	\begin{textblock*}{60mm}(0.3\paperwidth,0.25\paperheight)
		{\tiny \color{accentcolor}
		\includegraphics[width=60mm]{elements/fundamental domain.PNG}\\[2mm]
		}
	\end{textblock*}

	
\end{frame}


\section{Torus partition function}



\section{Modular invariance}











\end{document}